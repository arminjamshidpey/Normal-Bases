\section{Introduction}

For a finite Galois extension field $K/F$, with Galaois group
$G = \mathrm{Gal}(K/F)$, an element $\alpha \in K$ is called \emph{normal}
if the set of Galois conjugates of $\alpha$, $\{g\in G: g(\alpha)\}$, forms
a basis for $K$ as a vector space over $F$. The existence of normal element
for any finite Galois extension is classical, and constructive proofs are
provided in most abstract algebra texts (see, e.g., \cite{Wae70},
Section~8.11).
 
There are a wide range of applications of normal bases in finite
fields. Fast exponentiation or computing the action of the Frobenius and
point counting on elliptic curves are some of their applications. There are
also applications of normal elements in characteristic zero. For example,
for a given permutation lattice and appropriate Galois extension, a normal
basis is useful in computing the multiplicative invariants explicitly (see
\cite{Armin} for more details).

A number of algorithms are available for finding a normal element in
characteristic zero fields and finite fields.  Because of their immediate
applications in finite fields, algorithms for determining normal elements
in this case has seen the most activity.  A fast randomized algorithm for
determining a normal element in a finite field $\FF_{q^n}/\FF_q$, where
$\FF_{q^n}$ is the finite field with $q^n$ elements for any prime power
$q$ and integer $n>1$, is presented by \citeN{Giesbrecht}, with a cost of
$O(n^2+n\log q)$ operations in $\FF_q$.  A faster randomized algorithm is
introduced by \citeN{Kaltofen}, with a cost of $O(n^{1.8})$ operations in
$\FF_q$.  \citeN{LenstraNormal} introduced a deterministic algorithm to
construct a normal element which uses $O(n^{O(1)})$ operations in
$\FF_{q^n}/\FF_q$.  To the best of our knowledge, the algorithm of
\citeN{AugCam94} is the most efficient deterministic method, with a cost of
$O(n^3+n^2\log q)$ operations in $\FF_q$.

In characteristic zero, \citeN{SchSte93} gives an algorithm for finding a
normal basis of a number field over $\QQ$ with a cyclic Galois group which
requires $O(n^{O(1)})$ operations in $\QQ$.  \citeN{Pol94} gives an
algorithm for the more general case of finding a normal basis in an abelian
extension $K/F$ which requires $O(n^{O(1)})$ in $\F$.  More generally in
characteristic zero, for any Galois extension $K/F$ of degree $n$ with
Galois group given by a collection of $n$ matrices, \citeN{Girstmair} gives
an algorithm which requires $O(n^4)$ operations in $F$ to construct a
normal element in $K$.

In this paper we present a new randomized algorithm for finding a normal
element in case of abelian and metacyclic extensions which is subquadratic
in the degree of the extension. The idea behind the algorithm is similar to
ideas in \cite{Giesbrecht,Kaltofen}. Since one of the of the assumptions in
the results of this paper is the same, we state it here for future
references .

\begin{assumption}
  \label{assum}
  Let $K/F$ be a finite Galois extension presented as$F[x]/(f)$ for an
  irreducible polynomial $f\in F[x]$ of degree $n$, and
  $G = \mathrm{Gal}(K/F)$. Moreover, let $\alpha$ be an element of $K$ and
  the action of generators of $G$, such as $g$, on $K$ is given by its
  action on $\bar{x} = x \mod f.$ In other words, $g(\bar{x})$ is given for
  any generator $g$ of $G$.
\end{assumption}

\begin{remark}
  With the above assumption, suppose
  $$\beta = \sum_{0\leq i \leq n-1} a_i \bar{x}^i \in F[\bar{x}]$$. For any
  $g \in G$ and integer $t,$ we have
  \begin{equation}
    \label{rmk:comute}
    g^t(\beta) = \sum_{0\leq i \leq n-1} a_i g^t(\bar{x}^i) = \beta(g^t(\bar{x}))
  \end{equation}
  since $g$ acts trivially on $a_i$.
\end{remark}

Under the Assumption \ref{assum} we choose a random element $\alpha$ of
$K$. We use a known fact that, $\alpha$ is normal if
$M_G(\alpha) \in M_{n\times n}(K)$, an associated matrix to $\alpha$ (see
Section 2 for the definition), is invertible.  Afterwards we reduce the
invertiblity of $M_G(\alpha)$ to invertiblity of a random projection an
associated element $\osum{G}{K} \in K[G]$ which lies in $F[G]$, the group
ring of $G$ over $F$.

Section \ref{sec:pre} of this paper is devoted to provide the definitions
and preliminary discussions. In Section \ref{sec:osum} the orbit sum
problem is discussed and two algorithms are presented which can be applied
to compute projections of orbit sums.  Finally in the last section we do
what?!

%%% Local Variables:
%%% mode: latex
%%% TeX-master: "NormalBasisCharZero"
%%% End:
