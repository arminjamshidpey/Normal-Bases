\documentclass[sigconf]{acmart}

\usepackage{booktabs} % For formal tables
\usepackage{tikz}
\usepackage{mathdots}
\usepackage{esvect}
\usepackage{bbm}

\usepackage{xcolor}
\definecolor{darkgreen}{rgb}{0,.35,0}
\definecolor{darkblue}{rgb}{0,0,.5}
\definecolor{darkred}{rgb}{.6,0,0}

\newcommand{\smallskipback}{\vspace{-\smallskipamount}}
\newcommand{\medskipback}{\vspace{-\medskipamount}}
\newcommand{\bigskipback}{\vspace{-\bigskipamount}}

\usepackage{hyperref}
\hypersetup{pdfauthor={Mark Giesbrecht, Armin Jamshidpey, \'Eric Schost},
  pdftitle={Quadratic-Time Probabilistic Algorithms for Normal Bases},
  bookmarksnumbered=true,colorlinks,linkcolor=darkblue,
 citecolor=darkgreen, urlcolor=darkred}
% ]{hyperref}

\numberwithin{equation}{section}

\citestyle{acmauthoryear}

\newcommand{\osum}[2]{\alpha_{#1,#2}}
\newcommand{\osumcost}{O(n^{(3/4)\cdot \omega(4/3)})}
\newcommand{\osumcosttilde}{\tilde{O}(n^{(3/4)\cdot \omega(4/3)})}
\newcommand{\thecost}{\tilde{O}(\vert G \vert ^{(3/4)\cdot \omega(4/3)})}

\newcommand{\FF}{{\mathbb{F}}}
\newcommand{\xbar}{\xi}
\newcommand{\zbar}{\zeta}
\newcommand{\alg}{quadratic\,}

{
      \theoremstyle{acmplain}
      \newtheorem{assumption}{Assumption}
  }

{
      \theoremstyle{acmplain}
      \newtheorem{remark}{Remark}
  }


% Conference
\copyrightyear{2019} 
\acmYear{2019} 
\setcopyright{acmcopyright}
\acmConference[ISSAC '19]{International Symposium on Symbolic and Algebraic Computation}{July 15--18, 2019}{Beijing, China}
\acmBooktitle{International Symposium on Symbolic and Algebraic Computation (ISSAC '19), July 15--18, 2019, Beijing, China}
\acmPrice{15.00}
\acmDOI{10.1145/3326229.3326260}
\acmISBN{978-1-4503-6084-5/19/07}


\begin{document}
\title{Quadratic-Time Algorithms for Normal Elements}
%\titlenote{Produces the permission block, and
%  copyright information}
%\subtitle{Extended Abstract}
%\subtitlenote{The full version of the author's guide is available as
%  \texttt{acmart.pdf} document}


\author{Mark Giesbrecht
}
\affiliation{%
  \institution{Cheriton School of Computer Science
University of Waterloo}
}
\email{mwg@uwaterloo.ca}

\author{Armin Jamshidpey}
\affiliation{%
  \institution{Cheriton School of Computer Science
University of Waterloo}
}
\email{armin.jamshidpey@uwaterloo.ca}

\author{\'Eric Schost}
\affiliation{%
  \institution{Cheriton School of Computer Science
University of Waterloo}
}
\email{eschost@uwaterloo.ca}


% The default list of authors is too long for headers.
\renewcommand{\shortauthors}{Giesbrecht, Jamshidpey, Schost}

\newcommand{\F}{{\mathsf{F}}}
\newcommand{\K}{{\mathsf{K}}}

\newcommand{\NN}{{\mathbb{N}}}
\newcommand{\N}{{\mathbb{N}}}

\def\A{\mathbb{A}}
\def\H{\mathbb{H}}
\def\B{\mathbb{B}}
\def\Z{\mathbb{Z}}
\def\C{\mathbb{C}}
\def\Q{\mathbb{Q}}
\def\D{\mathbb{D}}
\newcommand{\QQ}{\mathbb{Q}}
\newcommand{\mat}[1]{\mathbf{\MakeUppercase{#1}}} % for a matrix

\begin{abstract}
  For any finite Galois field extension $\K/\F$, with Galois group $G =
  \mathrm{Gal}(\K/\F)$, there exists an element $\alpha \in \K$ whose
  orbit $G\cdot\alpha$ forms an $\F$-basis of $\K$. Such an $\alpha$
  is called a \emph{normal element} and $G\cdot\alpha$ is a
  \emph{normal basis}. We introduce a probabilistic
  algorithm for finding a normal element when $G$ is either a finite
  abelian or a metacyclic group. The algorithm is based on the fact
  that deciding whether a random element $\alpha \in \K$ is normal can
  be reduced to deciding whether $\sum_{\sigma \in G}
  \sigma(\alpha)\sigma \in \K[G]$ is invertible.  Our algorithm
  requires a quadratic number of operations in the size of $G$ for
  metacyclic $G$, and a slightly subquadratic number of operations for
  abelian $G$.
\end{abstract}

%
% The code below should be generated by the tool at
% http://dl.acm.org/ccs.cfm
% Please copy and paste the code instead of the example below. 
%
\begin{CCSXML}
<ccs2012>
<concept>
<concept_id>10010147.10010148.10010149.10010150</concept_id>
<concept_desc>Computing methodologies~Algebraic algorithms</concept_desc>
<concept_significance>500</concept_significance>
</concept>
</ccs2012>
\end{CCSXML}

%\ccsdesc[500]{Computer systems organization~Embedded systems}
%\ccsdesc[300]{Computer systems organization~Redundancy}
%\ccsdesc{Computer systems organization~Robotics}
%\ccsdesc[100]{Networks~Network reliability}
%
%
%\keywords{ACM proceedings, \LaTeX, text tagging}
\ccsdesc[500]{Computing methodologies~Algebraic algorithms}
\keywords{Normal bases; Galois groups; metacyclic groups; fast algorithms}


\maketitle

\section{Introduction}

For a finite Galois extension field $\K/\F$, with Galois group $G =
\mathrm{Gal}(\K/\F)$, an element $\alpha \in K$ is called
\emph{normal} if the set of its Galois conjugates $G \cdot \alpha =
\{\sigma\in G: \sigma(\alpha)\}$ forms a basis for $\K$ as a vector space over
$\F$. The existence of normal element for any finite Galois extension
is classical, and constructive proofs are provided in most algebra texts
(see, e.g., \cite{Lang}, Section 6.13).
%\cite{Wae70}, Section~8.11).
 
While there is a wide range of well-known applications of normal bases in
finite fields, such as fast exponentiation~\cite{GaGaPaSh00}, there also
exist applications of normal elements in characteristic zero.  For instance,
in multiplicative invariant theory, for a given permutation lattice and
related Galois extension, a normal basis is useful in computing the
multiplicative invariants explicitly~\cite{Armin}.

A number of algorithms are available for finding a normal element in
characteristic zero fields and finite fields.  Because of their immediate
applications in finite fields, algorithms for determining normal elements
in this case are most commonly seen.  A fast randomized algorithm for
determining a normal element in a finite field $\FF_{q^n}/\FF_q$, where
$\FF_{q^n}$ is the finite field with $q^n$ elements for any prime power $q$
and integer $n>1$, is presented by \citeN{Giesbrecht}, with a cost of
$O(n^2+n\log q)$ operations in $\FF_q$.  A faster randomized algorithm is
introduced by \citeN{Kaltofen}, with a cost of $O(n^{1.815}\log q)$
operations in $\FF_q$.  In the bit complexity model, Kedlaya and Umans showed
how to reduce the exponent of $n$ to $1.5+\varepsilon$ (for any
$\varepsilon > 0$), by leveraging their quasi-linear time algorithm for
{\em modular composition}~\cite{KeUm11}. \citeN{LenstraNormal} introduced a
deterministic algorithm to construct a normal element which uses $n^{O(1)}$
operations in $\FF_{q^n}/\FF_q$.  To the best of our knowledge, the
algorithm of \citeN{AugCam94} is the most efficient deterministic method,
with a cost of $O(n^3+n^2\log q)$ operations in $\FF_q$.

In characteristic zero, \citeN{SchSte93} gave an algorithm for finding
a normal basis of a number field over $\QQ$ with a cyclic Galois group
of cardinality $n$ which requires $n^{O(1)}$ operations in $\QQ$.
\citeN{Pol94} gives an algorithm for the more general case of finding
a normal basis in an abelian extension $\K/\F$ which requires
$O(n^{O(1)})$ in $\F$.  More generally in characteristic zero, for any
Galois extension $\K/\F$ of degree $n$ with Galois group given by a
collection of $n$ matrices, \citeN{Girstmair} gives an algorithm which
requires $O(n^4)$ operations in $\F$ to construct a normal element in
$\K$.

In this paper we present a new randomized algorithm for finding a normal
element for abelian and metacyclic extensions, with a runtime subquadratic
in the degree $n$ of the extension. The costs of all algorithms are
measured by counting \emph{arithmetic operations} in $\F$ at unit cost.
Questions related to the bit-complexity of our algorithms are challenging,
and beyond the scope of this paper.

Our main conventions are the following.
\begin{assumption}
  \label{assum}
  Let $\K/\F$ be a finite Galois extension presented as
  $\K=\F[x]/\langle F(x)\rangle$, for an irreducible polynomial $F\in
  \F[x]$ of degree $n$. Then,
  \begin{itemize}
  \item elements of $\K$ are written on the power basis $1,\bar x,\dots,\bar x^{n-1}$,
    where $\bar x = x \bmod F$;
  \item elements of $G$ are represented by their action on $\bar x$.
  \end{itemize}
\end{assumption}
In particular, for $\sigma \in G$ given by means of
$\gamma:=\sigma(\bar x) \in \K$, $\sigma$ being an $\F$-automorphism
implies that $\sigma(\beta)$ is equal to $\beta(\gamma)$.

Our algorithms combine techniques and ideas due
to~\cite{Giesbrecht,Kaltofen}: $\alpha \in \K$ is normal if and only
if the element $S_\alpha := \sum_{\sigma \in G} \sigma(\alpha)\sigma
\in \K[G]$ is invertible in the group algebra $\K[G]$. The algorithms
choose $\alpha$ at random; a generic choice is normal (so we expect
$O(1)$ random trials to be sufficient). However, writing down
$S_\alpha$ involves $\Theta(n^2)$ elements in $\F$, which precludes a
subquadratic runtime. Knowing $\alpha$, the algorithms use a
randomized reduction to a similar question in $\F[G]$, that amounts to
applying a random projection $\K\to\F$ to all entries of $S_\alpha$.
For that, we adapt algorithms from~\cite{Kaltofen} that were written
for Galois groups of finite fields.

Section \ref{sec:pre} of this paper is devoted to definitions and
preliminary discussions.  In Section \ref{sec:osum} two subquadratic-time
algorithms are presented for the randomize reduction of our main question
to invertibility testing in $\F[G]$, for respectively abelian and
metacyclic groups.  Finally, in Section \ref{sec:invertibility}, we show
that the latter problem can be solved in subquadratic time as well for the
families of groups we consider.

This paper is written from the point of view of obtaining improved
asymptotic complexity estimates. Since our main goal is to highlight the
exponent (in $n$) in our runtime analyses, costs are given using the soft-O
notation: $S(n)$ is in $\tilde{O}(T(n))$ if it is in
$O(T(n) \log(T(n))^c)$, for some constant $c$. Our algorithms make
extensive of known algorithms for polynomial and matrix arithmetic; in
particular, we use repeatedly the fact that polynomials of degree $d$ in
$\F[x]$, for any field $\F$ of characteristic zero, can be multiplied in
$\tilde{O}(n)$ operations in $\F$~\cite{ScSt71}. As a result, arithmetic 
operations $(+,\times,\div)$ in $\K$ can all be done using $\tilde{O}(n)$ 
operations in $\F$.

For matrix arithmetic, we will rely on some non-trivial results about
rectangular matrix multiplication initiated~\cite{LoRo83}. For $k \in
\mathbb{R}$, we denote by $\omega(k)$ a constant such that over any
ring, matrices of sizes $(n,n)$ by $(n,\lceil n^k \rceil)$ can be
multiplied in $O(n^{\omega(k)})$ ring operations (so $\omega(1)$ is
the usual exponent of square matrix multiplication, which we simply
write $\omega$).  The sharpest values known to date for most
rectangular formats are from~\cite{LeGall}; for $k=1$, the best known
value is $\omega \le 2.373$~\cite{LeGall14}. Over a field, we will
frequently use the fact that further matrix operations (determinant or
inverse) can be done in $O(n^\omega)$ base field operations.


%%% Local Variables:
%%% mode: latex
%%% TeX-master: "NormalBasisCharZero"
%%% End:

\section{Preliminaries}
\label{sec:pre}

One of the well-known proofs of the existence of a normal element for a
finite Galois extension \cite[Theorem 6.13.1]{Lang} suggests a randomized
algorithm for finding such an element. Assume $\K/\F$ is a finite Galois
extension with Galois group $G = \lbrace g_1 , \ldots , g_n \rbrace$. If
$x \in \K$ is a normal element, then
\begin{equation}
  \label{eq:fstrow}
  \sum_{j=1}^n 
  c_j g_j(x)=0, \,\,\, c_j \in \F 
\end{equation} 
implies $c_1 =\ldots=c_n = 0$. For each
$i \in \lbrace 1, \ldots , n\rbrace$, applying $g_i$ to equation
\eqref{eq:fstrow} yields
\begin{equation} \label{eq:otherrow} \sum_{j=1}^n c_j g_i g_j(x)=0.
\end{equation}
Using \eqref{eq:fstrow} and \eqref{eq:otherrow}, one can form a linear
system $\mat{M}_G(\alpha)\textbf{v} = \textbf{0}$ where, for $\alpha\in\K$,
\[
  \mat M_G(\alpha) =
  \begin{bmatrix}
    g_1 g_1(\alpha) & g_1 g_2(\alpha) & \cdots & g_1 g_n(\alpha) \\
    g_2 g_1(\alpha) & g_2 g_2(\alpha) & \cdots & g_2 g_n(\alpha) \\
    \vdots		& \vdots	& \vdots & \vdots \\
    g_n g_1(\alpha) & g_n g_2(\alpha) & \cdots & g_n g_n(\alpha) \\
  \end{bmatrix} \in M_n(\K).
\]
Classical proofs then proceed to show that there exists $\alpha \in \K$
with $\det(\mat M_G(\alpha))\neq 0$.
 
This approach can be used as the basis of a randomized algorithm for
finding a normal element: choose a random element $\alpha$ in $\K$ until we
find one such that $ \mat M_G(\alpha)$. A direct implementation computes
all the entries of the matrix and then use linear algebra to compute its
determinant; using fast matrix arithmetic this requires $O(n^\omega)$
operations in $\K$, that is $\tilde{O}(n^{\omega+1})$ operations in
$\F$. This is at least cubic in $n$, and only a minor improvement over the
previously best-known approach of \citeN{Girstmair}. The main contribution
of this paper is to show how to speed up this verification.
 
Before entering that discussion, we briefly discuss the probability that
$\alpha$ be a normal element: if we write
$\alpha = a_0 + \cdots + a_{n-1} \xbar^{n-1}$, the determinant of
$\mat M_G(\alpha)$ is a (not identically zero) homogeneous polynomial of
degree $n$ in $(a_0,\dots,a_{n-1})$. If the $a_i$'s are chosen uniformly at
random in a finite set $X \subset \F$, the Lipton-DeMillo-Schwartz-Zippel
implies that the probability that $\alpha$ be normal is at leat $1-n/|X|$.

If $G$ is cyclic then \cite{GatGie90} avoid the determinant computation by
computing the GCD of $S_\alpha(x) := \sum_{i = 0}^{n-1} g_i(\alpha)x^i$
and $x^n-1$. In effect, this amounts to testing whether $S_\alpha$ is
invertible in the group ring $\K[G]$, which is isomorphic to
$\K[x]/(x^n-1)$. This is a general fact: for any $G$, $\mat M_G(\alpha)$ is
the matrix of (left) multiplication by the orbit sum
$$S_\alpha:= \sum_{g \in G} g(\alpha)g \in \K[G],$$ and $\alpha$ being
normal is equivalent to $S_\alpha$ being a unit in $\K[G]$. This point of
view may make it possible to avoid linear algebra of size $n$ over $\K$,
but writing $S_\alpha$ itself still involves $\Theta(n^2)$ elements in
$\F$. The following lemma is the main new ingredient in our algorithm: it
gives a randomized reduction to testing whether a suitable projection of
$S_\alpha$ in $\F[G]$ is a unit.
 
%% This gives an idea to modify Algorithm \ref{Alg:Naive}. Instead of
%% writing $M_G(x)$ and test its invertiblity, we can write $\osum{G}{\K}$
%% and test if it is invertible in $\K[G]$. Although testing the
%% invertibility of $\osum{G}{\K}$ might be efficient in comparison to
%% computing the determinant of a matrix in $\K$, we prefer to do the
%% computations over $\F$ rather than $\K$. The following lemma comes handy
%% to pass the computations from $\K$ to $\F$.


\begin{lemma}
  \label{Lem:Proj}
  For $\alpha \in \K$, $\mat M_G(\alpha)$ is invertible if and only
  if $$\ell(\mat M_G(\alpha)) := [\ell(g_ig_j(\alpha))]_{ij} \in M_n(\F)$$
  is invertible, for a generic $\F$-linear projection $\ell: \K \to \F$.
\end{lemma}
\begin{proof}
  $(\Rightarrow)$ For a fixed $\alpha\in\K$, any entry of
  $\mat M_G(\alpha)$ can be written as
  \begin{equation}\label{Eq:PrimElm}
    \sum_{k= 0}^{n-1} a_{ijk}\xbar^k,
  \end{equation}
  and the corresponding entry in $\ell(\mat M_G(\alpha))$, for
  $\ell: \K \to \F$ can be written $\sum_{k= 0}^{n-1} a_{ijk}\ell_k$, with
  $\ell_k = \ell(\xbar^k)$. Replacing these $\ell_k$'s by indeterminates
  $L_k$'s, the determinant becomes a polynomial in
  $P \in \F[L_1, \ldots, L_n].$ Viewing $P$ in $\K[x_1, \ldots , x_n]$, we
  have $ P(1, \xbar, \ldots, \xbar^{n-1})$ $= \det(\mat M_G(\alpha))$,
  which is non-zero by assumption. Hence, $P$ is not identically zero, and
  the conclusion follows.
  
  $(\Leftarrow)$ Assume $\mat M_G(\alpha)$ is not invertible. Following the
  proof of \cite[Lemma 4]{Jam18}, we first show that there exists a
  non-zero $\boldsymbol{u} \in \F^n$ in the kernel of $\mat M_G(\alpha)$.
  
  The elements of $G$ act on rows of $\mat M_G(\alpha)$ entrywise and the
  action permutes the rows the matrix. Assume
  $\varphi : G \to \mathfrak{S}_n$ is the group homomorphism such that
  $g(\mat M_i) = \mat M_{\varphi(g)(i)}$ for all $i$, where $\mat M_i$ is
  the $i$-th row of $\mat M_G(\alpha)$.
  
  Since $\mat M_G(\alpha)$ is singular, there exists a non-zero
  $\boldsymbol{v} \in \K^n$ such that $\mat M_G(\alpha)\boldsymbol{v} = 0$;
  we choose $\boldsymbol{v}$ having the minimum number of non-zero
  entries. Let $i \in \lbrace 1, \ldots , n \rbrace$ such that
  $v_i \neq 0$. Define $\boldsymbol{u} = 1/v_i\boldsymbol{v}$. Then,
  $\mat M_G(\alpha)\boldsymbol{u} = 0,$ which means
  $\mat M_j \boldsymbol{u} = 0 $ for $j \in \lbrace 1, \ldots, n
  \rbrace$. For $g \in G$, we have
  $g(\mat M_j \boldsymbol{u}) = \mat M_{\varphi(g)(j)} g(\boldsymbol{u})=
  0.$ Since this holds for any $j$, we conclude that
  $\mat M_G(\alpha)g(\boldsymbol{u})= 0$, hence
  $g(\boldsymbol{u})-\boldsymbol{u}$ is in the kernel of
  $\mat M_G(\alpha)$. On the other hand since the $i$-th entry of
  $\boldsymbol{u}$ is one, the $i$-th entry of
  $g(\boldsymbol{u}) -\boldsymbol{u}$ is zero. Thus the minimality
  assumption on $\textbf{v}$ shows that
  $g(\boldsymbol{u}) -\boldsymbol{u} = 0$ and equivalently
  $g(\boldsymbol{u})=\boldsymbol{u}$ and hence $\boldsymbol{u} \in \F^n$.
  
  Now we show that $\ell(\mat M_G(\alpha))$ is not invertible for all
  choices of $\ell$. By Equation \eqref{Eq:PrimElm}, we can write
  $$\mat M_G(\alpha) = \sum_{j = 1}^n \mat M^{(j)} \xbar^j, \quad 
  \mat M^{j} \in M_{n}(\F) \text{~for all $j$}.$$ Now
  $\mat M_G(\alpha) \boldsymbol{u} =0$ yields
  $\mat M^{(j)}\boldsymbol{u} = 0$ for
  $j \in \lbrace 1, \ldots , n \rbrace$. Hence,
$$\sum_{j = 1}^n \mat M^{(j)} \ell_j \boldsymbol{u} = 0$$ for any 
$\ell_j$'s in $\F$, and $\ell(\mat M_G(\alpha))$ is not invertible for any~$\ell$.
\end{proof} 
Hence, our algorithm can be sketched as follows: choose random
$\alpha$ in $\K$ and $\ell: \K\to\F$, and let
$$s_{\alpha,\ell}:=\sum_{g \in G} \ell(g(\alpha))g \in \F[G].$$ The
matrix $\ell(\mat M_G (\alpha))$ is the multiplication matrix by
$s_{\alpha,\ell}$ in $\F[G]$, so once $s_{\alpha,\ell}$ is known, we
are left with testing whether it is a unit in $\F[G]$.
In the next two sections, we address the two questions highlighted above:
computing $s_{\alpha,\ell}$, and an invertibility test in $\F[G]$.


%%% Local Variables:
%%% mode: latex
%%% TeX-master: "NormalBasisCharZero"
%%% End:

\section{Computing projections of the orbit sum}
\label{sec:osum}

In this section we present algorithms to compute $s_{\alpha,\ell}$,
when $G$ is either abelian or metacyclic. We start by sketching our
ideas in simplest case, cyclic groups.  We will see that they follow
closely ideas used in \cite{KalSho98} over finite fields.

Suppose $G = \langle g \rangle$, so that given $\alpha$ in $\K$ and
$\ell: \K \to \F$, our goal is to compute
\begin{equation}
  \label{eq:cycproj}
  \ell(g^i(\alpha)), ~~\mbox{for}~ 0\leq i\leq n-1.
\end{equation}
\citeN{KalSho98} call this the \emph{automorphism projection problem} and
gave an algorithm to solve it in subquadratic time, when $g$ is the
$q$-power Frobenius $\mathbb{F}_{q^n} \to \mathbb{F}_{q^n}$.  The key idea in their
algorithm is to use the baby-steps/giant-steps technique: for a suitable
parameter $t$, the values in \eqref{eq:cycproj} can be rewritten as
\[
  (\ell \circ g^{tj})(g^i(\alpha)), ~~\mbox{for}~ 0 \leq j < m:=\lceil n/t
  \rceil ~\mbox{and}~ 0 \leq i <t.
\]
First, we compute all $G_i:=g^i(\alpha)$ for $0 \leq i <t$.  Then we compute
all $L_j:=\ell \circ g^{tj}$ for $0 \leq j <m$, where the $L_j$'s are
themselves linear mappings $\K \to \F$.  Finally, a matrix product yields
all values $L_j(G_i)$.

The original algorithm of \citeN{KalSho98} relies on the properties of the
Frobenius mapping to achieve subquadratic runtime. In our case, we cannot
apply these results directly; instead, we have to revisit the proofs
of~\citeN{KalSho98}, Lemmata 3, 4 and~8, now considering rectangular matrix
multiplication.  Our exponents involve the constant $\omega(4/3)$, for
which we have the upper bound $\omega(4/3) < 2.654$: this follows from the
upper bounds on $\omega(1.3)$ and $\omega(1.4)$ given by~\citeN{LeGall}, and
the fact that $k \mapsto \omega(k)$ is convex~\cite{LoRo83}. In particular,
$3/4 \cdot \omega(4/3) < 1.99$. Note also the inequality
$\omega(k) \ge 1+k$ for $k\ge 1$, since $\omega(k)$ describes products with
input and output size $O(n^{1+k})$.

%%%%%%%%%%%%%%%%%%%%%%%%%%%%%%%%%%%%%%%%%%%%%%%%%%%%%%%%%%%%

\subsection{Multiple automorphim evaluation and applications}

The key to the algorithms below is the remark following
Assumption~\ref{assum}, which reduces automorphism evaluation to
modular composition of polynomials.  Over finite fields, this idea goes back
to~\citeN{GaSh92}, where it was credited to Kaltofen.

For instance, given $g \in G$ (by means of $\gamma:=g(\xbar)$), we can
deduce $g^2 \in G$ (again, by means of its image at $\xbar$) as
$\gamma(\gamma)$; this can be done with $\tilde{O}(n^{(\omega+1)/2})$
operations in $\F$ using Brent and Kung's modular composition
algorithm~\cite{BrKu78}. The algorithms below describe similar operations
along these lines, involving several simultaneous evaluations.

\begin{lemma}
  \label{lem:modcom}
  Given $\alpha_1,\dots,\alpha_s$ in $\K$ and $g$ in $G =
  \mathrm{Gal}(\K/\F)$, with $s = O(\sqrt{n})$, we can compute
  $g(\alpha_1),\dots,g(\alpha_s)$ with $\tilde
  O(n^{(3/4)\cdot\omega(4/3)})$ operations in $\F$.
\end{lemma}
\begin{proof}
(Compare \cite[Lemma~3]{KalSho98}) As noted above, for $i\le s$,
  $g(\alpha_i) = \alpha_i(\gamma)$, with $\gamma := g(\xbar) \in \K$.
  Let $t := \lceil n^{3/4} \rceil$, $m:=\lceil n/t\rceil$, and rewrite $\alpha_1 , \ldots , \alpha_s$ as 
$$\alpha_i = \sum_{0 \leq j < m} a_{i,j}\xbar^{tj},$$ where the
  $a_{i,j}$'s are polynomials of degree less than $t$. The next step
  is to compute $\gamma_i := \gamma^i$, for $i = 0 , \ldots , t$.
  There are $t$ products in $\K$ to perform, so this amounts to
  $\tilde{O}(n^{7/4})$ operations in $\F$.

  Having $\gamma_i$'s in hand, one can form the matrix
  $\boldsymbol{\Gamma} := \left[ \Gamma_0 ~ \cdots ~ \Gamma_{t-1}
    \right]^T$, where each column $\Gamma_i$ is the coefficient vector
  of $\gamma_i$ (with entries in $\F$); this matrix has $t \in
  O(n^{3/4})$ rows and $n$ columns. We also form
  $$\mat A := \left[{A}_{1,0} \cdots {A}_{1,m-1} \cdots
    {A}_{s,0} \cdots {A}_{s,m-1}\right]^T,$$ where
  ${A}_{i,j}$ is the coefficient vector of $a_{i,j}$. This matrix 
  has $s m \in O(n^{3/4})$ rows and $t \in O(n^{3/4})$ columns.

  Compute $\mat B:=\mat A\, \boldsymbol{\Gamma}$; as per our
  definition of exponents $\omega(\cdot )$, this can be done in
  $O(n^{(3/4)\cdot \omega(4/3)})$ operations in $\F$, and the rows of this matrix
  give all $a_{i,j}(\gamma)$.  The last step to get all
  $\alpha_i(\gamma)$ is to write them as $\alpha_i(\gamma) = \sum_{0
    \leq j < m} a_{i,j}(\gamma) \Gamma_t^{j}.$ Using Horner's scheme,
  this takes $O(sm)$ operations in $\K$, which is $\tilde{O}(n^{7/4})$
  operations in $\F$. Since we pointed out that $\omega(3/4) \ge 7/4$,
  the leading exponent in all costs seen so far is
  $(3/4)\cdot\omega(4/3)$.
\end{proof}

\begin{lemma}\label{lem:selfcomp}
Given $\alpha$ in $\K$, $g_1, \ldots , g_{r}$ in $G =
\mathrm{Gal}(\K/\F)$ and positive integers $(s_1, \ldots s_r)$ such
that $\prod_{i = 1}^r s_i = O(\sqrt{n})$ and $r \in O(\log(n))$, all
  $$g_1^{i_1}\cdots g_r^{i_r}(\alpha) ,\quad \text{~for~} 0 \leq i_j
\leq s_j,\ 1 \leq j \leq r$$ can be computed in $\osumcosttilde$
operations in $\F$.
\end{lemma}
\begin{proof}
(Compare \cite[Lemma~4]{KalSho98}.) For $m=1,\dots,r$, suppose we have computed 
  $$G_{i_1,\dots,i_m}:=g_m^{i_m}\cdots g_1^{i_1}(\alpha)$$ for $0 \leq
  i_j \leq s_j$ if $1 \leq j < m$, and $0 \leq i_m < k_m,$ as well as
  the automorphism $\eta:={g_m}^{k_m}$ (by means of its value at $\xbar$, as per our convention).
  
 Then, we can obtain $G_{i_1,\dots,i_m}$ for $0 \leq i_j \leq s_j$ if $1
 \leq j < m$, and $0 \leq i_m < 2k_m$, by computing
 $\eta(G_{i_1,\dots,i_m})$, for all indices $i_1,\dots,i_m$ available
 to us, that is, $0 \leq i_j \leq s_j$ if $1 \leq j < m$, and $0 \leq
 i_m < k_m$. This can be carried out using $\osumcosttilde$ operations
 in $\F$ by applying Lemma \ref{lem:modcom}. Prior to entering the
 next iteration, we also compute $\eta^2$ by means of one modular
 composition, whose cost is negligible. 

 Using the above doubling method for $g_m$, we have to do $O(\log
 s_m)$ steps, for a total cost of $\osumcosttilde$ operations in $\F$.  We
 repeat this procedure for $m=1,\dots,r$; since $r$ is in $O(\log(n))$,
 the cost remains $\osumcosttilde$.
\end{proof}

We now present dual versions of the previous two lemmas (our
reference~\cite{KalSho98} also had such a discussion). Seen as an
$\F$-linear map, the operator $g:\alpha \mapsto g(\alpha)$ admits a
transpose, which maps an $\F$-linear form $\ell:\K\to\F$ to the
$\F$-linear form $\ell \circ g: \alpha \mapsto \ell(g(\alpha))$.  The
{\em transposition principle}~\cite{KaKiBs88,CaKaYa89} implies that if
a linear map $\F^N \to \F^M$ can be computed in time $T$, its
transpose can be computed in time $T+O(N+M)$. In particular, given $s$
linear forms $\ell_1,\dots,\ell_s$ and $g$ in $G$, transposing
Lemma~\ref{lem:modcom} shows that we can compute $\ell_1 \circ
g,\dots,\ell_s \circ g$ in time $\osumcosttilde$. The following lemma
sketches the construction.

\begin{lemma}
  \label{lem:modcomT}
  Given $\F$-linear forms $\ell_1,\dots,\ell_s:\K\to \F$ and $g$ in $G =
  \mathrm{Gal}(\K/\F)$, with $s = O(\sqrt{n})$, we can compute
  $\ell_1\circ g,\dots,\ell_s \circ g$ in time $\tilde
  O(n^{{3}/{4}\omega({4}/{3})})$.
\end{lemma}
\begin{proof}
  Given $\ell_i$ by its values on the power basis $1,\xbar,\dots,\xbar^{n-1}$, $\ell_i \circ g$ is represented by its values at
  $1,\gamma,\dots,\gamma^{n-1}$, with $\gamma := g(\xbar)$. 

  Let $t,m$ and $\gamma_0,\dots,\gamma_t$ be as in the proof of
  Lemma~\ref{lem:modcom}. Next, compute the ``giant steps''
  $\gamma_t^j = \gamma^{tj}$, $j=0,\dots,m-1$ and for $i=1,\dots,s$
  and $j=0,\dots,m-1$, deduce the linear forms $L_{i,j}$ defined by
  $L_{i,j}(\alpha) := \ell_i(\gamma^{tj}\alpha)$ for all $\alpha$ in
  $\K$. Each of them can be obtained by a {\em transposed
    multiplication} in time $\tilde{O}(n)$~\cite[Section~4.1]{Shoup},
  so that the total cost thus far is $\tilde{O}(n^{7/4})$.

  Finally, multiply the $(sm \times n)$ matrix with entries the
  coefficients of all $L_{i,j}$ (as rows) by the $(n \times t)$ matrix with
  entries the coefficients of $\gamma_0,\dots,\gamma_{t-1}$ (as columns) to
  obtain all values $\ell_i(\gamma^j)$, for $i=1,\dots,s$ an
  $j=0,\dots,n-1$.  This can be accomplished with
  $O(n^{(3/4)\cdot\omega(4/3)})$ operations in~$\F$.
\end{proof}

From this, we deduce the transposed version of Lemma~\ref{lem:selfcomp},
whose proof follows the same pattern.

\begin{lemma}
  \label{lem:transmodcomp}
  Given $\ell:\K\to F$, $g_1, \ldots , g_{r}$ in $G = \mathrm{Gal}(\K/\F)$
  and positive integers $(s_1, \ldots s_r)$ such that
  $\prod_{i = 1}^r s_i = O(\sqrt{n})$ and $r \in O(\log(n))$, all linear
  maps
  \[
    \ell \circ g_1^{i_1}\cdots g_r^{i_r} ,\quad \text{~for~} 0 \leq i_j
    \leq s_j,\ 1 \leq j \leq r
  \]
  can be computed in $\osumcosttilde$ operations in $\F$.
\end{lemma} 
\begin{proof}
  We proceed as in Lemma~\ref{lem:selfcomp}. For $m=1,\dots,r$, assume
  we know $L_{i_1,\dots,i_m}:=\ell \circ (g_1^{i_1}\cdots g_m^{i_m}),$
  for $0 \leq i_j \leq s_j$ if $1 \leq j < m$, and $0 \leq i_m < k_m.$
  Using the previous lemma, we compute all $L_{i_1,\dots,i_m} \circ
  {g_m}^{k_m},$ which gives us $L_{i_1,\dots,i_m}$ for indices $0 \le
  i_m < 2k_m$. The cost analysis is as in Lemma~\ref{lem:selfcomp}.
\end{proof}

%% At this point we have enough tools to see how the computation is
%% done in the cyclic case. Moreover, we can use the above lemmas to
%% give an algorithm for a more general case, namely the abelian
%% case. With a little bit more work we can state an algorithm which
%% solves the automorphism projection problem when $G = \lbrace a^ib^j
%% \rbrace$ where $m \leq n$. As an specific case this solves the
%% problem for metacyclic groups.

%%%%%%%%%%%%%%%%%%%%%%%%%%%%%%%%%%%%%%%%%%%%%%%%%%%%%%%%%%%%

\subsection{Abelian Groups}
\label{ssec:proj_abelian}

The first main result in this section is the following proposition.
Assume $G$ is an abelian group presented as 
\[
  \langle g_1, \ldots , g_r: g_{1}^{e_1} = \cdots = g_{r}^{e_r} = 1
  \rangle,
\]
where $ e_i \in \mathbb{N}$ is the order of $g_i$ and $n =
e_1 \cdots e_r$.  Without loss of generality, we assume $e_i \ge 2$ for
all $i$, so that $r$ is in $O(\log n)$. Elements of the group
algebra $\F[G]$ are written as commutative polynomials
$\sum_{i_1,\dots,i_r} c_{i_1,\dots,i_r} {g_1}^{e_1} \cdots {g_r}^{e_r}$,
with $0\le i_j < e_j$ for all~$j$.


\begin{proposition}\label{prop:abelian}
  Suppose that $G$ is abelian, with notation as above. For $\alpha$ in $\K$ and $\ell:\K\to\F$, 
  $s_{\alpha,\ell} \in \F[G]$ is computable using  $\osumcosttilde$
  operations in $\F$.
\end{proposition}
\begin{proof}
Our goal is to compute
\begin{equation}\label{eq:abelian}
  \ell (g_1^{i_1},  \ldots, g_r^{i_r}(\alpha)), \, 1 \leq j \leq r, 0 \leq i_j \leq e_j,
\end{equation}
where $\ell$ is an $\F$-linear projection $\K\to \F$.  For $ 1\leq i
\leq r$, define $s_i:=\lceil\sqrt{e_i \rceil}$. As we sketched in the
cyclic case, the elements in \eqref{eq:abelian} can be expressed as
$L_{j_1,\cdots, j_r} (G_{i_1,\dots,i_r})$, 
for $1\leq m \leq r, 0\leq i_m < s_m, 0 \leq j_m < s_m$.
Here, $L_{j_1\cdots j_r} :=\ell \circ (g_1^{s_1j_1} \cdots
g_r^{s_rj_s})$ are linear projections presented as row vectors and
$G_{i_1,\dots,i_r}:=g_1^{i_1} \cdots g_r^{i_r}(\alpha)$ are field
elements presented as column vectors. All elements in
\eqref{eq:abelian} can be computed with the following steps, the sum of whose 
costs proves the proposition.

\smallskip\noindent \textbf{Step 1.} Apply Lemma \ref{lem:selfcomp} to get 
$$G_{i_1,\dots,i_r}=g_1^{i_1} \cdots g_r^{i_r}(\alpha), \,\, 1\leq m \leq r, 0\leq i_m < s_m,$$
with cost $\osumcosttilde$.

\smallskip\noindent\textbf{Step 2.} Compute all $g_i^{s_i}$, $i=1,\dots,r$;
this involves $O(\log(n))$ modular compositions, so the cost is negligible
compared to that of Step~1.

\smallskip\noindent\textbf{Step 3.} Use Lemma \ref{lem:transmodcomp} to compute 
$$L_{j_1,\cdots,j_r} = \ell \circ (g_1^{s_1j_1} \cdots
g_r^{s_rj_s}), \,\, 1\leq m \leq r, 0 \leq j_m < s_m,$$
with cost $\osumcosttilde$

\smallskip\noindent\textbf{Step 4.} Multiply the matrix with rows the
coefficients of all $L_{j_1,\cdots,j_r}$ by the matrix with columns
the coefficients of all $G_{i_1,\dots,i_r}$; this yields all required
values, as pointed out above. We can compute this  product in
$O(n^{(1/2)\cdot\omega(2)})$ operations in $\F$, which is in $\osumcost$.
\end{proof}

%%%%%%%%%%%%%%%%%%%%%%%%%%%%%%%%%%%%%%%%%%%%%%%%%%%%%%%%%%%%

\subsection{Metacyclic Groups}

A group $G$ is metacyclic if it has a normal cyclic subgroup $H$ such that
$G/H$ is cyclic; for instance, any group with a squarefree order is
metacyclic. See \cite[p.~88]{Johnson} or \cite[p.~334]{Curtis} for more
background. A metacyclic group can always be presented~as
\begin{equation}
  \label{eq:metacyclic}
  \langle \sigma,\tau: \sigma^m = 1,  \tau^s = \sigma^t, \tau^{-1}\sigma \tau = \sigma^r \rangle,
\end{equation}
for some integers $m,t,r,s$, with $r,t \leq m$ and
$r^s = 1 \bmod t, rt = t \bmod m$. For example, the dihedral group
$$D_{2m} = \langle \sigma,\tau: \sigma^m =1, \tau^2 = 1, \tau^{-1}
\sigma \tau = \sigma^{m-1} \rangle, $$ is metacyclic, with
$s=2$. Generalized quaternion groups, which can be presented as
$$Q_m = \langle \sigma,\tau: \sigma^{2m} =1, \tau^2 = \sigma^m,
\tau^{-1} \sigma \tau = \sigma^{2m-1} \rangle,$$ are metacyclic, with
$s=2$ as well.

Using the notation of~\eqref{eq:metacyclic}, $n=|G|$ is equal to $ms$, and
all elements in a metacyclic group can be presented uniquely as either
\begin{equation}\label{pres1}
\{\sigma^i \tau^j,\,\,\, 0\leq i \leq m-1,\ 0\leq j \leq s-1\}  
\end{equation}
or
\begin{equation}\label{pres2}
\{ \tau^j\sigma^i,\,\,\, 0\leq i \leq m-1,\ 0\leq j \leq s-1\}.
\end{equation}
Accordingly, elements in the group algebra $\F[G]$ can be written as 
either 
$$\sum_{\substack{i <m\\ j< s}} c_{i,j} \sigma^i \tau^j \quad\text{or}\quad
\sum_{\substack{i <m\\ j< s}} c'_{i,j} \tau^j \sigma^i.$$
Conversion between the two representations involves no operation in $\F$,
using the commutation relation $\sigma^k \tau^c = \tau^c \sigma^{kr^c}$
for $k,c \ge 0$.

\begin{proposition}
  Suppose that $G$ is metacyclic, with notation as above. For $\alpha$
  in $\K$ and $\ell:\K\to\F$, $s_{\alpha,\ell} \in \F[G]$ is
  computable using $\osumcosttilde$ operations in $\F$.
\end{proposition}
\begin{proof}
  Suppose first that $s \le m$; then, we use the
  presentation~\eqref{pres1} of the elements of $G$. Take $\alpha$ in
  $\K$ and $\ell:\K \to \F$; the goal is to compute
  $\ell(\sigma^i\tau^j (\alpha))$, for all $0\leq i < m$ and $0 \leq j
  <s.$ This is accomplished with the following steps.

\smallskip\noindent\textbf{Step 1.} Apply Lemma \ref{lem:selfcomp} to compute
$$G_{i,j} := \sigma^i\tau^j(\alpha),\ 0\leq i < \lceil \sqrt{m/s}
\rceil,\ 0 \leq j < s.$$ Note that
$\lceil \sqrt{m/s} \rceil s \leq \lceil \sqrt{sm} \rceil \in O(\sqrt n)$,
so we are under the assumptions of the lemma. This takes $\osumcosttilde$
operations in~$\F$.

\smallskip\noindent\textbf{Step 2.} Compute
$\sigma^{\lceil \sqrt{m/s} \rceil}$, in $O(\log(n))$ modular compositions
in degree $n$. The cost is no more than that of Step 1.

\smallskip\noindent\textbf{Step 3.} Compute
$$L_k := \ell \circ \sigma^{k\lceil \sqrt{m/s} \rceil}, \,\, 0\leq k <
\lceil \sqrt{sm}\rceil$$ using Lemma \ref{lem:transmodcomp}.  This takes
$\osumcosttilde$ operations in~$\F$.

\smallskip\noindent\textbf{Step 4.} At this point, we compute all
$$ L_k(s_{i,j}) = \ell(\sigma^{k\lceil \sqrt{m/s} \rceil + i}\tau
^j(\alpha)),$$ for $0\le k < \lceil \sqrt{sm}\rceil$,
$0\le i< \lceil \sqrt{m/s}\rceil$ and $0 \leq j < s;$ these are precisely
the values we needed.

This can be carried out by multiplying the matrix with rows the
coefficients of all $L_k$ by the matrix with columns the coefficients of
all $G_{i,j}$; this yields all required values, as pointed out above. There
are $O(\sqrt{sm})=O(\sqrt{n})$ linear forms $L_k$'s, and $O(\sqrt{n})$
field elements $G_{i,j}$'s, so we can compute this product in
$O(n^{(1/2)\cdot\omega(2)})$ operations in $\F$, which is $\osumcost$.

This concludes the proof in the case $s \le m$. When $m \le s$, use the
presentation~\eqref{pres2} of the elements of $G$ and proceed as above.
\end{proof}

%% We note the above algorithm works for a class of groups which includes metacyclic case. Since if $G$ is metacyclic, 
%% $H = \langle \sigma \rangle \unlhd G$ and $G/H = \langle \tau H \rangle$, then both $\tau \sigma \tau^{-1}$ and 
%% $\tau^{-1} \sigma \tau$ belong to $H$. This implies elements of $G$ can be presented either as $\sigma^i\tau^j$
%% or $\tau^j\sigma^i$. Thus without loss of generality we can assume the $\textrm{Ord}(\tau) \leq \textrm{Ord}(\sigma)$.

%% Similar to the abelian case the final output of the above algorithm is a $\lceil \sqrt{n} \rceil \times \lceil \sqrt{n} \rceil $
%% matrix and $l(\osum{G}{K})$ can be calculated in the same way. Hence we have proved the following proposition.
 
%% \begin{proposition}
%% Suppose Assumption \ref{assum} holds and $G$ is a metacyclic group. $l(\osum{K}{G}) \in F[G]$ is computable using $\thecost$ 
%% operations in $F$.
%% \end{proposition}

%%% Local Variables:
%%% mode: latex
%%% TeX-master: "NormalBasisCharZero"
%%% End:

\section{Arithmetic in the Group Algebra}
\label{sec:invertibility}

In this section we consider the problems of invertibility testing and
division in $\F[G]$: given elements $\beta,\eta$ in $\F[G]$, for a
field $\F$ and a group $G$, determine whether $\beta$ is a unit in
$\F[G]$, and if so, compute $\beta^{-1}\eta$. We focus on two
particular families of polycyclic groups, namely abelian and
metacyclic groups $G$; as well as being necessary in our application
to normal bases, we believe these problems are of independent
interest.

Since we are in characteristic zero, Wedderburn's theorem implies the
existence of an $\F$-algebra isomorphism (which we will refer to as a
Fourier Transform)
\[
  \F[G] \to M_{d_1}(D_1) \times \dots \times M_{d_r}(D_r),
\]
where all $D_i$'s are division algebras over $\F$. If we were working
over $\F=\C$, all $D_i$'s would simply be $\C$ itself.  A natural
solution to test the invertibility of $\beta \in \F[G]$ would then be
to compute its Fourier transform and test whether all its components
$\beta_1 \in M_{d_1}(\C),\dots,\beta_r \in M_{d_r}(\C)$ are
invertible. This boils down to linear algebra over $\C$, and takes
$O(d_1^\omega + \cdots + d_r^\omega)$ operations.  Since $d_1^2 +
\cdots + d_r^2 = n$, with $n=|G|$, this is $O(n^{\omega/2})$
operations in $\C$.

However, we do not wish to make such a strong assumption as $\F=\C$. Since
we measure the cost of our algorithms in $\F$-operations, the direct
approach that embeds $\F[G]$ into $\C[G]$ does not make it possible to
obtain a subquadratic cost in general. If, for instance, $\F=\Q$ and $G$ is
cyclic of order $n=2^k$, computing the Fourier Transform of $\beta$
requires we work in a degree $n/2$ extension of $\Q$, implying a quadratic
runtime.

In this section, we give algorithms for the problems of invertibility
testing and division for the two particular families of polycyclic
groups mentioned so far, namely abelian and metacyclic. For the former,
starting from a suitable presentation of $G$, we give a softly
linear-time algorithm to find an isomorphic image of $\beta \in \F[G]$
in a product of $\F$-algebras of the form $\F[z]/\langle
P_i(z)\rangle$, for certain polynomials $P_i \in \F[z]$ (recovering
$\beta$ from its image is softly-linear time as well). Not only does
this allow us to test whether $\beta$ is invertible, this also makes
it possible to find its inverse in $\F[G]$ (or to compute products in
$\F[G]$) in softly-linear time (we are not aware of previous results
of this kind).

For metacyclic groups, we rely on the block-Hankel structure of the
matrix of multiplication by $\beta$. Through structured linear algebra
algorithms, this allows us to solve both problems (invertibility and
division) in subquadratic (albeit not softly-linear time) time.

%%%%%%%%%%%%%%%%%%%%%%%%%%%%%%%%%%%%%%%%%%%%%%%%%%%%%%%%%%%%

\subsection{Abelian groups}

Because an abelian group is a product of cyclic groups, the group
algebra $\F[G]$ of such a group is the tensor product of cyclic
algebras. Using this property, given an element $\beta$ in $\F[G]$,
our goal in this section is to determine whether $\beta$ is a unit,
and if so to compute expressions such as $\beta^{-1} \eta$, for
$\eta$ in $\F[G]$.

The previous property implies that $\F[G]$ admits a description of the
form $\F[x_1,\dots,x_t]/\langle x_1^{n_1}-1,\dots,x_t^{n_t}-1\rangle$,
for some integers $n_1,\dots,n_t$. The complexity of arithmetic
operations in an $\F$-algebra such as $\A:=\F[x_1,\dots,x_t]/\langle
P_1(x_1),\dots,P_t(x_t)\rangle$ is difficult to pin down precisely. For
general $P_i$'s, the cost of multiplication in $\A$ is known to be
$O(\dim(\A)^{1+\varepsilon})$, for any $\varepsilon >
0$~\citep[Theorem~2]{LiMoSc09}. From this it may be possible to deduce
similar upper bounds on the complexity of invertibility test or division,
following~\citep{DaMMMScXi06}, but this seems non-trivial.

Instead, we give an algorithm with softly linear runtime, that uses
the factorization properties of cyclotomic polynomials and Chinese
remaindering techniques to transform our problem into that of
invertibility test or division in algebras of the form $\F[z]/\langle
P_i(z) \rangle$, for various polynomials $P_i$.  \cite{Pol94} also
discusses the factors of algebras such as $\F[x_1,\dots,x_t]/\langle
x_1^{n_1}-1,\dots,x_t^{n_t}-1\rangle$, but the resulting algorithms
are different (and the cost of the \citeauthor{Pol94}'s
\citeyearpar{Pol94} algorithm is only known to be polynomial in
$n=|G|$).

\smallskip

\noindent{\bf Tensor product of two cyclotomic rings: coprime orders.}
The following proposition will be the key to foregoing multivariate
polynomials, and replacing them by univariate ones.  Let $m,m'$ be two
coprime integers and define
$$\mathbbm{h}:=\F[x,x']/\langle \Phi_{m}(x), \Phi_{m'}(x')\rangle,$$
where for $i \ge 0$, $\Phi_i$ is the cyclotomic polynomial of order
$i$. In what follows, $\varphi$ is Euler's totient function, so that
$\varphi(i) = \deg(\Phi_i)$ for all~$i$.
\begin{lemma}
  There exists an $\F$-algebra isomorphism $\gamma: \mathbbm{h} \to
  \F[z]/\langle\Phi_{mm'}(z)\rangle$ given by $xx' \mapsto z$.  Given
  $\Phi_m$ and $\Phi_{m'}$, $\Phi_{mm'}$ can be computed in time
  $\tilde{O}(\varphi(mm'))$; given these polynomials, one can
  apply $\gamma$ and its inverse to any input using
  $\tilde{O}(\varphi(mm'))$ operations in~$\F$.
\end{lemma}
\begin{proof}
  Without loss of generality, we prove the first claim over $\Q$; the
  result over $\F$ follows by scalar extension. In the field \sloppy
  $\Q[x,x']/\langle \Phi_{m}(x), \Phi_{m'}(x')\rangle$, $xx'$ is
  cancelled by $\Phi_{mm'}$. Since this polynomial is irreducible, it
  is the minimal polynomial of $xx'$, which is thus a primitive
  element for $\Q[x,x']/\langle \Phi_{m}(x),
  \Phi_{m'}(x')\rangle$. This proves the first claim.

  For the second claim, we first determine the images of $x$ and $x'$
  by $\gamma$. Start from a B\'ezout relation $am+ a'm'=1$, for some
  $a,a'$ in $\Z$.  Since $x^m = {x'}^{m'}=1$ in $\mathbbm{h}$, we
  deduce that $\gamma(x)=z^{u}$ and $\gamma(x') = z^{v}$, with $u:=am
  \bmod mm'$ and $v:=a'm' \bmod mm'$. To compute $\gamma(P)$, for some
  $P$ in $\mathbbm{h}$, we first compute $P(z^u, z^v)$, keeping all
  exponents reduced modulo $mm'$. This requires no arithmetic
  operations and results in a polynomial $\bar P$ of degree less than
  $mm'$, which we eventually reduce modulo $\Phi_{mm'}$ (the latter is
  obtained by the composed product algorithm of~\cite{BoFlSaSc06} in
  quasi-linear time).  By~\citep[Theorem~8.8.7]{BacSha96}, we have the
  bound $s \in O(\varphi(s) \log(\log(s)))$, so that $s$ is in
  $\tilde{O}(\varphi(s))$. Thus, we can reduce $\bar P$ modulo
  $\Phi_{mm'}$ in $\tilde{O}(\varphi(mm'))$ operations, establishing
  the cost bound for $\gamma$.

  Conversely, given $Q$ in $\F[z]/\langle\Phi_{mm'}(z)\rangle$, we obtain
  its preimage by replacing powers of $z$ by powers of $xx'$, reducing all
  exponents in $x$ modulo $m$, and all exponents in $x'$ modulo $m'$.  We
  then reduce the result modulo both $\Phi_m(x)$ and $\Phi_{m'}(x')$.  By
  the same argument as above, the cost is softly linear in $\varphi(mm')$.
\end{proof}

\noindent{\bf Extension to several cyclotomic rings.}  The natural
generalization of the algorithm above starts with pairwise distinct
primes $\boldsymbol{p}=(p_1,\dots,p_t)$, non-negative exponent
$\boldsymbol{c}=(c_1,\dots,c_t)$ and variables
$\boldsymbol{x}=(x_1,\dots,x_t)$ over $\F$. Now, we define
$$\H:=\F[x_1,\dots,x_t]/\langle
\Phi_{{p_1}^{c_1}}(x_1),\dots,\Phi_{{p_t}^{c_t}}(x_t)\rangle;$$ when
needed, we will write $\H$ as
$\H_{\boldsymbol{p},\boldsymbol{c},\boldsymbol{x}}$. Finally, we let
$\mu:={p_1}^{c_1}\cdots {p_t}^{c_t}$; then, the dimension $\dim(\H)$ is
$\varphi(\mu)$.

\begin{lemma}\label{lemma:distinctP}
 There exists an $\F$-algebra isomorphism $\Gamma: \H \to
 \F[z]/\langle\Phi_{\mu}(z)\rangle$ given by $x_1 \cdots x_t \mapsto
 z$.  One can apply $\Gamma$ and its inverse to any input using
 $\tilde{O}(\dim(\H))$ operations in $\F$.
\end{lemma}
\begin{proof}
  We proceed iteratively. First, note that the cyclotomic polynomials
  $\Phi_{{p_i}^{c_i}}$ can all be computed in time $O(\varphi(\mu))$. 
  The isomorphism
  $\gamma: \F[x_1,x_2]/\langle \Phi_{{p_1}^{c_1}}(x_1),
  \Phi_{{p_2}^{c_2}}(x_2)\rangle \to \F[z]/\langle
  \Phi_{{p_1}^{c_1}{p_2}^{c_2}}(z)\rangle$
given in the previous paragraph extends coordinate-wise to an
  isomorphism
  $$\Gamma_1: \H \to \F[z,x_3,\dots,x_t]/\langle
  \Phi_{{p_1}^{c_1}{p_2}^{c_2}}(z),\Phi_{{p_3}^{c_3}}(x_3),\dots,\Phi_{{p_t}^{c_t}}(x_t)\rangle.$$
  By the previous lemma, $\Gamma_1$ and its inverse can be applied to
  any input in time $\tilde{O}(\varphi(\mu))$. Iterate this process
  another $t-2$ times, to obtain $\Gamma$ as a product
  $\Gamma_{t-1} \circ \cdots \circ \Gamma_1$. Since $t$ is logarithmic 
  in $\varphi(\mu)$, the proof is complete.
\end{proof}

\noindent{\bf Tensor product of two prime-power cyclotomic rings, same $p$.}~In the following two paragraphs, we discuss the opposite situation as
above: we now work with cyclotomic polynomials of prime power
orders for a common prime $p$. As above, we start with two such polynomials.

Let thus $p$ be a prime. The key to the following algorithms is the
lemma below.  Let $c,c'$ be positive integers, with $c \ge
c'$, and let $x,y$ be indeterminates over $\F$. Define
\begin{align}
\mathbbm{a}&:=\F[x]/\Phi_{p^c}(x),  \\
\mathbbm{b}&:=\F[x,y]/\langle \Phi_{p^c}(x), \Phi_{p^{c'}}(y)\rangle = \mathbbm{a}[y]/\Phi_{p^{c'}}(y).
\end{align}
Note that $\mathbbm{a}$ and $\mathbbm{b}$ have respective dimensions
$\varphi(p^c)$ and $\varphi(p^c) \varphi(p^{c'})$.
\begin{lemma}
  There is an $\F$-algebra isomorphism $\theta: \mathbbm{b} \to
  \mathbbm{a}^{\varphi(p^{c'})}$ such that one can apply $\theta$ or
  its inverse to any inputs using $\tilde{O}(\dim(\mathbbm{b}))$ operations in $\F$.
\end{lemma}
\begin{proof}
  Let $\xbar$ be the residue class of
  $x$ in $\A$. Then, in $\mathbbm{a}[y]$, $\Phi_{p^{c'}}(y)$ factors as
  $$\Phi_{p^{c'}}(y) =\prod_{\substack{1 \le i\le p^{c'}-1\\ \gcd(i,p)
      =1}} (y-\rho_i),$$ with $\rho_i:={\xbar}^{i p^{c-c'}}$ for all
  $i$.  Even though $\mathbbm{a}$ may not be a field, the Chinese
  Remainder theorem implies that $\mathbbm{b}$ is isomorphic to
  $\mathbbm{a}^{\varphi(p^{c'})}$; the isomorphism is given by
  $$\begin{array}{cccc}
    \theta: & \mathbbm{b} & \to & \mathbbm{a} \times \cdots \times \mathbbm{a}, \\
    & P & \mapsto& (P(\xbar,\rho_1),\dots,P(\xbar,\rho_{\varphi(p^{c'})}).
  \end{array}$$
  In terms of complexity, arithmetic operations $(+,-,\times)$ in
  $\mathbbm{a}$ can all be done in $\tilde{O}(\varphi(p^c))$ operations
  in $\F$. Starting from $\rho_1 \in \mathbbm{a}$, all other roots
  $\rho_i$ can then be computed in $O(\varphi(p^{c'}))$ operations in
  $\mathbbm{a}$, that is, $\tilde{O}(\dim(\mathbbm{b}))$
  operations in $\F$. 
  
Applying $\theta$ and its inverse is done by means of fast evaluation
and interpolation~\citep[Chapter~10]{vzGathen13} in $\tilde{O}(\varphi(p^{c'}))$
operations in $\mathbbm{a}$, that is, $\tilde{O}(\deg(\mathbbm{b}))$ operations in $\F$
(the algorithms do not require that $\mathbbm{a}$ be a field).
\end{proof}

\smallskip\noindent{\bf Extension to several cyclotomic rings.}
Let $p$ be as before, and consider now non-negative integers
$\boldsymbol{c}=(c_1,\dots,c_t)$ and variables $\boldsymbol{x}=(x_1,\dots,x_t)$. We
define the $\F$-algebra
$$\A:=\F[x_1,\dots,x_t]/\langle \Phi_{p^{c_1}}(x_1), \dots,
\Phi_{p^{c_t}}(x_t)\rangle,$$ which we will sometimes write
$\A_{p,\boldsymbol{c},\boldsymbol{x}}$ to make the dependency on $p$
and the $c_i$'s clear. Up to reordering the $c_i$'s, we can assume
that $c_1 \ge c_i$ holds for all $i$, and define as before
$\mathbbm{a}:=\F[x_1]/\Phi_{p^{c_1}}(x_1)$.

\begin{lemma}\label{lemma:A}
  There exists an $\F$-algebra isomorphism $\Theta: \A \to
  \mathbbm{a}^{\dim(\A)/\dim(\mathbbm{a})}$. This isomorphism and its
  inverse can be applied to any inputs using $\tilde{O}(\dim(\A))$
  operations in $\F$.
\end{lemma}
\begin{proof}
Without loss of generality, we can assume that all $c_i$'s are non-zero
(since for $c_i=0$, $\Phi_{p^{c_i}}(x_i)=x_i-1$,
so $\F[x_i]/\langle \Phi_{p^{c_i}}(x_i) \rangle = \F$).
We proceed iteratively. First, rewrite $\A$ as
$$\A=\mathbbm{a}[x_2,x_3,\dots,x_t]/\langle \Phi_{p^{c_2}}(x_2), \Phi_{p^{c_3}}(x_3), \dots,
\Phi_{{p_t}^{c_t}}(x_t)\rangle.$$ 
The isomorphism 
$\theta: \mathbbm{a}[x_2]/\Phi_{p^{c_2}}(x_2) \to \mathbbm{a}^{\varphi(p^{c_2})}$
introduced in the previous paragraph extends coordinate-wise
to an isomorphism 
$$\Theta_1: \A \to (\mathbbm{a}[x_3,\dots,x_t]/\langle
\Phi_{p^{c_3}}(x_3), \dots,
\Phi_{p^{c_t}}(x_t)\rangle)^{\varphi(p^{c_2})};$$ $\Theta_1$ and its
inverse can be evaluated in quasi-linear time $\tilde{O}(\dim(\A))$.
We now work in all copies of $\mathbbm{a}[x_3,\dots,x_t]/\langle
\Phi_{p^{c_3}}(x_3), \dots, \Phi_{p^{c_t}}(x_t)\rangle$ independently,
and apply the procedure above to each of them. Altogether we have
$t-1$ such steps to perform, giving us an isomorphism
$$\Theta = \Theta_{t-1} \circ \cdots \circ \Theta_1:
\A \to
\mathbbm{a}^{\varphi(p^{c_2}) \cdots \varphi(p^{c_t})}.$$
The exponent can be rewritten as $ \dim(\A)/\dim(\mathbbm{a})$, as claimed.
In terms of complexity, all $\Theta_i$'s and their inverses can be computed
in quasi-linear time $\tilde{O}(\dim(\A))$, and we do $t-1$ of them,
where $t$ is $O(\log(\dim(\A)))$. 
\end{proof}

\noindent{\bf Decomposing certain $p$-group algebras.}  The prime $p$
and indeterminates $\boldsymbol{x}=(x_1,\dots,x_t)$ are as before; we now consider
positive integers $\boldsymbol{b}=(b_1,\dots,b_t)$, and the $\F$-algebra
\[
\begin{array}{ccl}
\B&:=&\F[x_1,\dots,x_t]/\langle x_1^{p^{b_1}}-1,\dots,x_t^{p^{b_t}}-1\rangle\\$$
&=& \F[x_1]/\langle x_1^{p^{b_1}}-1 \rangle \otimes \cdots \otimes \F[x_t]/\langle x_t^{p^{b_t}}-1 \rangle.
\end{array}
\]
If needed, we will write $\B_{p,\boldsymbol{b},\boldsymbol{x}}$ to make the dependency
on $p$ and the $b_i$'s clear. This is the $\F$-group algebra
of $\Z/p^{b_1}\Z \times \cdots \times \Z/p^{b_t}\Z$.

\begin{lemma}\label{lemma:alg}
  There exists a positive integer $N$, non-negative integers
  $\boldsymbol{c}=(c_1,\dots,c_N)$ and  an
  $\F$-algebra isomorphism 
  $$\Lambda: \B \to \D= \F[z]/\langle \Phi_{p^{c_1}}(z) \rangle \times \cdots \times \F[z]/\langle \Phi_{p^{c_N}}(z)\rangle.$$
  One can apply the isomorphism and its inverse to any 
  input using $\tilde{O}(\dim(\B))$ operations in $\F$.
\end{lemma}
\begin{proof}
For $i \le t$, we have the factorization
$$x_i^{p^{b_i}}-1 = \Phi_1(x_i) \Phi_p(x_i) \Phi_{p^2}(x_i) \cdots
\Phi_{p^{b_i}}(x_i);$$ note that $\Phi_1(x_i)=x_i-1$.  The factors may
not be irreducible, but they are pairwise coprime, so that we have a
Chinese Remainder isomorphism
\[
  \lambda_i: \F[x_i]/\langle x_i^{p^{b_i}}-1 \rangle \to \F[x_i]/\langle \Phi_1(x_i)\rangle
  \times \cdots \times  \F[x_i]/\langle \Phi_{p^{b_i}}(x_i)\rangle.
\]
Together with its inverse, this can be computed  
in $\tilde{O}(p^{b_i})$ operations in $\F$~\citep[Chapter~10]{vzGathen13}. By distributivity of the tensor
product over direct products, 
this gives an $\F$-algebra isomorphism
$$\lambda: \B \to \prod_{c_1=0}^{b_1} \cdots \prod_{c_t=0}^{b_t} \A_{p,\boldsymbol{c},\boldsymbol{x}},$$
with $\boldsymbol{c}=(c_1,\dots,c_t)$. Together with its inverse, 
$\lambda$ can be computed in $\tilde{O}(\dim(\B))$ operations in $\F$.
Composing with the result in Lemma~\ref{lemma:A}, this gives
us an isomorphism
$$\Lambda: \B \to \D:=\prod_{c_1=0}^{b_1} \cdots \prod_{c_t=0}^{b_t}
\mathbbm{a}_{\boldsymbol{c}}^{D_{\boldsymbol{c}}},$$ where
$\mathbbm{a}_{\boldsymbol{c}} = \F[z]/\langle \Phi_{p^c}(z)\rangle$,
with $c =\max(c_1,\dots,c_t)$ and $D_{\boldsymbol{c}} =
\dim(\A_{t,\boldsymbol{c},\boldsymbol{x}})/\dim(\mathbbm{a}_{\boldsymbol{c}})$. As
before, $\Lambda$ and its inverse can be computed in quasi-linear time
$\tilde{O}(\dim(\B))$.
\end{proof}
As for $\B$, we will write $\D_{p,\boldsymbol{b},\boldsymbol{x}}$ if needed; it is
well-defined, up to the order of the factors.

\smallskip

\noindent{\bf Main result.} Let $G$ be an abelian group.  We can write
the elementary divisor decomposition of $G$ as $G = G_1 \times \cdots
\times G_s$, where each $G_i$ is of prime power order $p_i^{a_i}$, for
pairwise distinct primes $p_1,\dots,p_s$, so that $n = |G|$ writes $n
= p_1^{a_1} \cdots p_s^{a_s}$. Each $G_i$ can itself be written as a
product of cyclic groups, $G_i = G_{i,1} \times \cdots \times
G_{i,t_i}$, where the factor $G_{i,j}$ is cyclic of order
${p_i}^{b_{i,j}}$, with $b_{i,1} \le \cdots \le b_{i,t_i}$; this is
the invariant factor decomposition of $G_i$, with $b_{i,1} + \cdots +
b_{i,t_i} = a_i$.

We henceforth assume that generators
$\gamma_{1,1},\dots,\gamma_{s,t_s}$ of respectively
$G_{1,1},\dots,G_{s,t_s}$ are known, and that elements of $\F[G]$ are
given on the power basis in $\gamma_{1,1},\dots,\gamma_{s,t_s}$. Were
this not the case, given arbitrary generators $g_1,\dots,g_r$ of $G$, with
orders $e_1,\dots,e_r$, a brute-force solution would factor each $e_i$
(factoring $e_i$ takes $o(e_i)$ bit operations on a standard RAM), so
as to write $\langle g_i \rangle$ as a product of cyclic groups of
prime power orders, from which the required decomposition follows.

\begin{proposition}
  Given $\beta \in \F[G]$, written on the power basis
  $\gamma_{1,1},\dots,\gamma_{s,t_s}$, one can test if $\beta$ is a
  unit in $\F[G]$ using $\tilde{O}(n)$ operations in $\F$.
  If it is the case, given $\eta$ in $\F[G]$, one can compute
  $\beta^{-1} \eta$ in the same asymptotic runtime.
\end{proposition}
In view of
Lemma~\ref{Lem:Proj-bis}, Proposition~\ref{prop:polycyclic} and the
claim on the cost of invertibility testing prove the first part of
Theorem \ref{thm:main}; the second part of this proposition will allow
us to prove Theorem~\ref{thm:main2} in the next section.

\smallskip

The proof of the proposition occupies the rest of this paragraph.
From the factorization $G = G_1 \times \cdots \times G_s$, we deduce
that the group algebra $\F[G]$ is the tensor product $\F[G_1] \otimes
\cdots \otimes \F[G_s]$. Furthermore, the factorization $G_i = G_{i,1}
\times \cdots \times G_{i,t_i}$ implies that $\F[G_i]$ is isomorphic,
as an $\F$-algebra, to
$$\F[x_{i,1},\dots,x_{i,t_i}]/\left \langle
x_{i,1}^{p_i^{b_{1}}}-1,\dots,x_{i,t_i}^{p_i^{b_{i,t_i}}}-1\right\rangle
=\B_{p_i,\boldsymbol{b}_i,\boldsymbol{x}_i},$$ with $\boldsymbol{b}_i
= (b_{i,1},\dots,b_{i,t_i})$ and $\boldsymbol{x}_i =
(x_{i,1},\dots,x_{i,t_i})$. Given $\beta$ on the power basis in
$\gamma_{1,1},\dots,\gamma_{s,t_s}$, we obtain its image $B$ in
$\B_{p_1,\boldsymbol{b}_1,\boldsymbol{x}_1} \otimes \cdots \otimes
\B_{p_s,\boldsymbol{b}_s,\boldsymbol{x}_s}$ simply by renaming
$\gamma_{i,j}$ as $x_{i,j}$, for all $i,j$.

For $i \le s$, by Lemma~\ref{lemma:alg}, there exist integers
$c_{i,1},\dots,c_{i,N_i}$ such that
$\B_{p_i,\boldsymbol{b}_i,\boldsymbol{x}_i}$ is isomorphic to an
algebra $\D_{p_i, \boldsymbol{b}_i, z_i}$, with factors 
$\F[z_i]/\langle \Phi_{{p_i}^{c_{i,j}}}(z_i) \rangle$.
By distributivity of the tensor product over direct products, we
deduce that $\B_{p_1,\boldsymbol{b}_1,\boldsymbol{x}_1} \otimes \cdots
\otimes \B_{p_s,\boldsymbol{b}_s,\boldsymbol{x}_s}$ is isomorphic to
the product of algebras
 \begin{equation}\label{eq:prod}
\text{\small $\prod$}_{\boldsymbol{j}}~ \F[z_1,\dots,z_s]/
\langle \Phi_{{p_1}^{c_{1,j_1}}}(z_1),\dots, \Phi_{{p_s}^{c_{s,j_s}}}(z_s) \rangle,   
 \end{equation}
for indices $\boldsymbol{j}=(j_1,\dots,j_s)$, with
$j_1 =1,\dots,N_1,\dots,j_s=1,\dots,N_s$;
call $\Gamma$ the isomorphism. Given $B$ in $\B_{p_1,\boldsymbol{b}_1,\boldsymbol{x}_1} \otimes
\cdots \otimes \B_{p_s,\boldsymbol{b}_s,\boldsymbol{x}_s}$,
Lemma~\ref{lemma:alg} also implies that $B':=\Gamma(B)$ can be
computed in softly linear time $\tilde{O}(n)$ (apply the isomorphism
corresponding to $\boldsymbol{x}_1$ coordinate-wise with respect to
all other variables, then deal with $\boldsymbol{x}_2$, etc).
The codomain in~\eqref{eq:prod} is the product of all $\H_{\boldsymbol{p},\boldsymbol{c}_{\boldsymbol{j}},\boldsymbol{z}}$,
with 
$$\boldsymbol{p}=(p_1,\dots,p_s),\quad \boldsymbol{c}=(c_{1,j_1},\dots,c_{s,j_s}),\quad \boldsymbol{z}=(z_1,\dots,z_s).$$
Apply Lemma~\ref{lemma:distinctP} to all 
$\H_{\boldsymbol{p},\boldsymbol{c}_{\boldsymbol{j}},\boldsymbol{z}}$ to obtain
an $\F$-algebra isomorphism
$$\Gamma': \text{\small $\prod$}_{\boldsymbol{j}}~
\H_{\boldsymbol{p},\boldsymbol{c}_{\boldsymbol{j}},\boldsymbol{z}} \to
\text{\small $\prod$}_{\boldsymbol{j}} ~\F[z]/\langle
\Phi_{d_{\boldsymbol{j}}}(z) \rangle,$$ for certain integers
$d_{\boldsymbol{j}}$. The lemma implies that given $B'$,
$B'':=\Gamma'(B')$ can be computed in softly linear time
$\tilde{O}(n)$ as well. Invertibility of $\beta \in \F[G]$ is
equivalent to $B''$ being invertible, that is, to all its components
being invertible in the respective factors $\F[z]/\langle
\Phi_{d_{\boldsymbol{j}}}(z) \rangle$. Invertibility in such an
algebra can be tested in softly linear time by applying the fast
extended GCD algorithm~\citep[Chapter~11]{vzGathen13}, so the first
part of the proposition follows.

Given $\eta$ in $\F[G]$, we can similarly compute its image $H''$ in
$\text{\small $\prod$}_{\boldsymbol{j}} ~\F[z]/\langle
\Phi_{d_{\boldsymbol{j}}}(z) \rangle,$ with the same asymptotic
runtime as for $\beta$. If we suppose $\beta$ (and thus $B''$)
invertible, division in each $\F[z]/\langle
\Phi_{d_{\boldsymbol{j}}}(z) \rangle$ takes softly linear time in the
degree $\phi_{d_{\boldsymbol{j}}}$; as a result, we obtain ${B''}^{-1}
H''$ in time $\tilde{O}(n)$.  One can finally invert all
isomorphisms we applied, in order to recover $\beta^{-1} \eta$ in
$\F[G]$; this also takes time $\tilde{O}(n)$. Summing all costs,
this establishes the second part of the proposition.

%%%%%%%%%%%%%%%%%%%%%%%%%%%%%%%%%%%%%%%%%%%%%%%%%%%%%%%%%%%%

\subsection{Metacyclic Groups}

In this subsection, we study the invertibility and division problems
for a metacyclic group $G$. A group $G$ is metacyclic if it has a
normal cyclic subgroup $H$ such that $G/H$ is cyclic: this is the case
$r=2$ in the definition we gave of polycyclic groups. For instance,
any group with a squarefree order is metacyclic, see
\citep[p.~88]{Johnson} or \citep[p.~334]{Curtis} for more
background. 

For such groups, we will use a standard specific notation, rather than
the general one introduced in~\eqref{eq:polycyclicgrp} for arbitrary
polycyclic ones: we will write $(\sigma,\tau)$ instead of $(g_1,g_2)$
and $(m,s)$ instead of $(e_1,e_2)$. Then, a metacyclic group $G$ can
be presented~as
\begin{equation}
  \label{eq:metacyclic}
  \langle \sigma,\tau: \sigma^m = 1,  \tau^s = \sigma^t, \tau^{-1}\sigma \tau = \sigma^u \rangle,
\end{equation}
for integers $m,t,u,s$, with $u,t \leq m$ and $u^s = 1 \bmod t$, $ut =
t \bmod m$. For example, the dihedral group
$$D_{2m} = \langle \sigma,\tau: \sigma^m =1, \tau^2 = 1, \tau^{-1}
\sigma \tau = \sigma^{m-1} \rangle, $$ is metacyclic, with
$s=2$. Generalized quaternion groups, which can be presented as
$$Q_m = \langle \sigma,\tau: \sigma^{2m} =1, \tau^2 = \sigma^m,
\tau^{-1} \sigma \tau = \sigma^{2m-1} \rangle,$$ are metacyclic, with
$s=2$ as well. Using the notation of~\eqref{eq:metacyclic}, $n=|G|$ is
equal to $ms$, and all elements in a metacyclic group can be presented
uniquely as either
\begin{equation}\label{pres1}
\{\sigma^i \tau^j,\,\,\, 0\leq i \leq m-1,\ 0\leq j \leq s-1\}  
\end{equation}
or
\begin{equation}\label{pres2}
\{ \tau^j\sigma^i,\,\,\, 0\leq i \leq m-1,\ 0\leq j \leq s-1\}.
\end{equation}
Accordingly, elements in the group algebra $\F[G]$ can be written as 
either 
$$\sum_{\substack{i <m\\ j< s}} c_{i,j} \sigma^i \tau^j
\quad\text{or}\quad \sum_{\substack{i <m\\ j< s}} c'_{i,j} \tau^j
\sigma^i.$$ Conversion between the two representations involves no
operation in $\F$, using the commutation relation $\sigma^k \tau^c =
\tau^c \sigma^{ku^c}$ for $k,c \ge 0$.

To test invertibility in $\F[G]$, a possibility would be to rely on
the Wedderburn decomposition of $\F[G]$, but the structure of group
algebras of metacyclic groups is not straightforward to exploit; see
for instance \citep[\S 47]{Curtis} for algebraically closed $\F$, or,
when $\F=\Q$,~\citep{Decomposition} for dihedral and quaternion
groups. Instead, we will highlight the structure of the multiplication
matrices in $\F[G]$.

Take $\beta$ in $\F[G]$. In eq.~\eqref{eqdef:M}, we introduced the
matrix $\mat M_\F(\beta)$ of left multiplication by $\beta$ in
$\F[G]$, where columns and rows were indexed using an arbitrary ordering
of the group elements. We will now reorder the rows and columns of
$\mat M_\F(\beta)$ using the two presentations of $G$ seen in
\eqref{pres1} and~\eqref{pres2}, in order to highlight its block structure.
In what follows, for non-negative integers $a,b,c$, we will write
$\beta_{a,b,c}$ for the coefficient of $\tau^a \sigma^b \tau^c$ in the
expansion of $\beta$ on the $\F$-basis of $\F[G]$.

We first rewrite $\mat M_\F(\beta)$ by reindexing its columns
by 
$$\begin{bmatrix}
(\sigma^0 \tau^0)^{-1} &  \cdots & (\sigma^{m-1} \tau^0)^{-1} \cdots& (\sigma^0 \tau^{s-1})^{-1} &  \cdots & (\sigma^{m-1} \tau^{s-1})^{-1} 
\end{bmatrix}$$
and its rows by
$$\begin{bmatrix}
\tau^0 \sigma^0 & \cdots & \tau^0 \sigma^{m-1} & \cdots& \tau^{s-1} \sigma^0 & \cdots&  \tau^{s-1} \sigma^{m-1}
\end{bmatrix}.
$$ This matrix displays a $s \times s$
block structure. Each block has itself size $m \times m$; for $1 \le
u,v \le s$ and $1 \le a,b \le m$, the entry of index $(a,b)$ in the
block of index $(u,v)$ is the coefficient of $\tau^u \sigma^a \sigma^b
\tau^v$ in $\beta$, that is, $\beta_{u,a+b,v}$. In other words,
all blocks are Hankel matrices.

Using the algorithm of~\cite{BoJeMoSc17} (see
also~\citealt[Appendix~A]{EbGiGiSVi07}), this structure allows us to
solve a system such as $\mat M_\F(\beta) \boldsymbol x = \boldsymbol
y$ in Las Vegas time $\tilde{O}(s^{\omega-1} n)$ (or raise an error if
there is no solution). In addition, if the right-hand side is zero and
$\mat M_\F(\beta)$ is not invertible, the algorithm returns a non-zero
kernel element.
This last remark allows us to test whether $\beta$ is invertible in
Las Vegas time $\tilde{O}(s^{\omega-1} n)$; if so, given the
coefficient vector $\boldsymbol y$ of some $\eta$ in $\F[G]$, we can
compute $\beta^{-1} \eta$ in the same asymptotic runtime.

It is also possible to reorganize the rows and columns of 
 $\mat M_\F(\beta)$, using indices
$$\begin{bmatrix}
(\tau^0 \sigma^0)^{-1} & \cdots & (\tau^0 \sigma^{m-1})^{-1} & \cdots& (\tau^{s-1} \sigma^0)^{-1} & \cdots&  (\tau^{s-1} \sigma^{m-1})^{-1}
\end{bmatrix}$$
for its columns and 
$$\begin{bmatrix}
\sigma^0 \tau^0 &  \cdots & \sigma^{m-1} \tau^0 \cdots& \sigma^0 \tau^{s-1} &  \cdots & \sigma^{m-1} \tau^{s-1}
\end{bmatrix}
$$ for its rows. The resulting matrix has an $m \times m$ block
structure, where each $s\times s$ block is Hankel. As a result, it
allows us to solve the problems above, this time using
$\tilde{O}(m^{\omega-1} n)$ operations in $\F$.  Since we have either
$s\le \sqrt n$ or $m \le \sqrt n$, this implies the following.

\begin{proposition}
  Given $\beta \in \F[G]$, one can test if $\beta$ is a unit in
  $\F[G]$ using $\tilde{O}(n^{(\omega+1)/2})$ operations in $\F$.  If
  it is the case, given $\eta$ in $\F[G]$, one can compute $\beta^{-1}
  \eta$ in the same asymptotic runtime.
\end{proposition}
Combined with Proposition~\ref{prop:polycyclic}, the former statement
provides the last part of the proof of Theorem \ref{thm:main}.






%%% Local Variables:
%%% mode: latex
%%% TeX-master: "NormalBasisCharZero"
%%% End:

%\section{Basis Conversion}

\textbf{Normal to Power basis}
Suppose $G = \lbrace g_1, \ldots, g_n \rbrace$, $\alpha$ is a normal element of $K$ and $u = \begin{bmatrix}u_1 & \ldots & u_n \end{bmatrix} \in F^n$ is given as the coefficient vector of $u$ in the normal basis.
In order to write $u $ in the power basis one can compute 
\begin{equation}\label{eq:norm2pwrmat}
\left[\begin{array}{c|c|c}
\overline{g_1(\alpha)} & \cdots & \overline{g_n(\alpha)} 
\end{array}\right]\cdot 
\begin{bmatrix}
u_1 & \cdots & u_n
\end{bmatrix}^t,
\end{equation}
where $\overline{g_i(\alpha)}$ is the coefficient vector of $g_i(\alpha)$. The result of \eqref{eq:norm2pwrmat} is the coefficient vector of
\begin{equation}\label{eq:norm2pwrpol}
\sum_{i = 1}^n u_i g_i(\alpha).
\end{equation}
Note that \eqref{eq:norm2pwrmat} shows that conversion from normal to power basis is transpose of the problem of computing the projections of 
conjugates of $\alpha$ which is solved in Section \label{sec:osum}. Here we present an explicit algorithm to solve this problem i.e. to compute
\eqref{eq:norm2pwrpol} for abelian and metacyclic $G$. Before stating the algorithm for basis conversion, we present the last
ingredient needed. 

\begin{lemma}
Under the Assumption \ref{assum}, suppose $g_1, \ldots g_r \in G$ and $h_{i_1 \cdots i_r} \in K$ for $0 \leq j \leq r, \, 
1 \leq i_j \leq s_j, \, \prod_{j = 1}^r s_j \leq \vert \lceil \sqrt{G} \rceil \vert$ are given. the elements 
$$g_r^{i_r}\cdots g_1^{i_1}(h_{i_1 \cdots i_r}) 0 \leq j \leq r, \, 
1 \leq i_j \leq s_j, \, \prod_{j = 1}^r s_j \leq \vert \lceil \sqrt{G} \rceil \vert$$
can be computed using $\thecost$ operations over $F$.
\end{lemma}

\begin{proof}
Let $h_{i_1 \cdots i_r}^{(0 \cdots 0)} = h_{i_1 \cdots i_r}$ and $$g_r^{l_r} \cdots g_1^{l_1}(h_{i_1 \cdots i_r}^{(u_1 \cdots u_r)}) = h_{i_1 \cdots i_r}^{(u_1+l_1 \cdots u_r+l_r)}$$ for $0 \leq j \leq r$ and $0 \leq l_j$.
Assume $0 \leq t < r$ is fixed and $$h_{i_1 \cdots i_r}^{(i_1 \cdots i_{t-1}0 \cdots 0)}, \, 0 \leq j < t, \, 
1 \leq i_j \leq s_j,$$ is already computed. We compute 
$$h_{i_1 \cdots i_r}^{(i_1 \cdots i_{t}0 \cdots 0)}, \, 0 \leq j \leq t, \, 1 \leq i_j \leq s_j,$$
by iteratively applying the following method. In $m$-th iteration we apply $g_t^{2^{m-1}}$ to $h_{i_1 \cdots i_r}^{(i_1 \cdots i_{t}0 \cdots 0)}$ for $j \leq t-1$, $0 \leq i_j \leq s_j$ and 
$$i_t \in \left(\cup_{k = 1}^{\infty} \lbrace (2k-1)2^{m-1}, \ldots , (2k)2^{m-1}-1 \rbrace\right) \cap \lbrace 0 , \ldots , s_t \rbrace $$
At each iteration we have to compute $O(\vert \lceil \sqrt{G} \rceil \vert)$ modular composition which can be done by applying
Lemma \ref{lem:modcom} and $g^{2^i}$ can be computed by applying \ref{lem:selfcom}. This gives the claimed complexity.
\end{proof}

%%% Local Variables:
%%% mode: latex
%%% TeX-master: "NormalBasisCharZero"
%%% End:



\citestyle{acmauthoryear}
\bibliographystyle{ACM-Reference-Format}
\bibliography{NormalBasisCharZero} 

\end{document}

%%% Local Variables:
%%% mode: latex
%%% TeX-master: t
%%% End:
